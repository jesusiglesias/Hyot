% Document type
\documentclass[12pt,a4paper, twoside]{report}

% Packages
\usepackage[usenames,dvipsnames,svgnames,table]{xcolor}
\usepackage{amsmath}
\usepackage{amssymb}
\usepackage{appendix}
\usepackage{caption} 			% Caption
\usepackage{color}
\usepackage{dirtytalk} 			% Quote
\usepackage{eso-pic}
\usepackage{eurosym}
\usepackage{fancyhdr} 			% Header
\usepackage{float}
\usepackage{fontspec}
\usepackage{graphicx} 			% Graphics
\usepackage{hyperref} 			% References
\usepackage{listings} 			% Write code
\usepackage{lmodern}
\usepackage{longtable}
\usepackage{multirow, array} 
\usepackage{pdfcolmk}
\usepackage[spanish]{babel} 		% Spanish
\usepackage{slantsc}
\usepackage{subcaption} 			% Subfigure
\usepackage[svgnames]{xcolor}
\usepackage{tikz}
\usepackage{url}
\usepackage[utf8]{inputenc}
\usepackage{xcolor,colortbl}
\usepackage{wrapfig} 			% Position of image

% Colours
\definecolor{gray96}{gray}{.96}
\definecolor{gray75}{gray}{.75}
\definecolor{gray45}{gray}{.45}

% Code format
\lstset{ frame=Ltb,
framerule=0pt,
aboveskip=0.5cm,
framextopmargin=3pt,
framexbottommargin=3pt,
framexleftmargin=0cm,
framesep=0pt,
rulesep=.4pt,
backgroundcolor=\color{gray96},
rulesepcolor=\color{black},
%
stringstyle=\ttfamily,
showstringspaces = false,
basicstyle=\small\ttfamily,
commentstyle=\color{gray45},
keywordstyle=\bfseries,
%
numbers=left,
numbersep=15pt,
numberstyle=\tiny,
numberfirstline = false,
breaklines=true,
}

% Minimizar fragmentado de listados
\lstnewenvironment{listing}[1][]
{
	\lstset{#1}\pagebreak[0]
}{
	\pagebreak[0]
}

\lstdefinestyle{consola} {
	basicstyle=\scriptsize\ttfamily,
	backgroundcolor=\color{gray96},
}

% Index colour
\hypersetup{colorlinks=true, linkcolor=blue, urlcolor=blue}
%\renewcommand\thesection{\arabic{section}}

% Margin
\usepackage[a4paper, tmargin=2cm, bmargin=4cm, right=1.5cm, left=2cm]{geometry}

% Commands
\newcommand{\university}{UNIVERSIDAD AUTÓNOMA DE MADRID}
\newcommand{\faculty}{ESCUELA POLITÉCNICA SUPERIOR}
\newcommand{\dpto}{DEPARTAMENTO DE INGENIERÍA INFORMÁTICA}
\newcommand{\master}{Máster Universitario en Ingeniería Informática}
\newcommand{\tfm}{TRABAJO FIN DE MÁSTER}
\newcommand{\titleTFM}{DESARROLLO DE UN SISTEMA DE TRAZABILIDAD EN ENTORNOS IOT MEDIANTE HYPERLEDGER}
\newcommand{\jesus}{IGLESIAS GARCÍA, JESÚS}
\newcommand{\david}{ARROYO GUARDEÑO, DAVID}
% Blank page
\newcommand{\blankpage}{
				\newpage
				\thispagestyle{empty}
				\mbox{}
				\newpage
				}
% List of tables
\renewcommand{\listtablename}{Índice de tablas}
% Line spacing
\renewcommand{\baselinestretch}{1.2}
% Depth in index (section, subsection, etc.)
\setcounter{tocdepth}{2}
% Annexes without number page
\let\plainappendixpage\appendixpage
\makeatletter
\renewcommand{\appendixpage}{%
  \begingroup
  \let\ps@plain\ps@empty
  \plainappendixpage
  \endgroup}

% DOCUMENT
\begin{document}

	% Start roman numerals
	\pagenumbering{roman}
	
	% TITLE 
	\begin{titlepage}
		\begin{center}
			\vspace*{0.5in}	

			\Large{\textsc{\textbf{\university}}}\\	
			\vspace*{0.2in}
			\large{\textsc{\faculty}}\\
			\vspace*{0.5in}
			
			\begin{figure}[htb]
				\begin{center}
					\includegraphics[width=17cm, height=3.5cm]{Imagenes/eps_uam}
				\end{center}
			\end{figure}
			\vspace*{0.6in}
			\Large{\textsc{\tfm}}\\
			\vspace*{0.35in}
			\Large{\textbf{\titleTFM}} \\
			\vspace*{0.8in}
			\large{\textsc{\master}}\\
			\vspace*{0.1in}
			\rule{100mm}{0.3mm}\\ % Línea horizontal
			\vspace*{0.5in}
			\normalsize{\textbf{AUTOR:}} \normalsize{\jesus}  \\
			\vspace*{0.1in}
			\normalsize{\textbf{TUTOR:}} \normalsize{\david} \\
			\vspace*{0.4in}
			\large{{\color{red}XX} 2018} \\
		\end{center}
	\end{titlepage}
	
	\blankpage
	
	% TITLE 
	\begin{titlepage}
		\begin{center}
			\vspace*{1.5in}	
			
			\large{\textbf{\titleTFM}} \\
			\vspace*{0.8in}
			\normalsize{AUTOR:} \normalsize{\jesus}  \\
			\vspace*{0.05in}
			\normalsize{TUTOR:} \normalsize{\david} \\
			\vspace*{0.8in}
			\small{\textsc{\dpto}}\\
			\vspace*{0.02in}
			\small{\textsc{\faculty}}\\
			\vspace*{0.02in}
			\small{\textsc{\university}}\\	
			\vspace*{0.02in}
			\small{{\color{red}XX} 2018} \\
		\end{center}
	\end{titlepage}
	
	\blankpage
	
	% Acknowledgments
	\chapter*{Agradecimientos}
	\thispagestyle{empty}
	\vspace{1cm}
	\begin{flushright}
			\textit{Dedicado a mis padres por el gran esfuerzo que han hecho durante toda mi vida para apoyarme durante la consecución de todas las metas propuestas tanto personales como profesionales.} 
			
		\vspace{1cm}	
		
			\textit{Agradecer en especial a David Arroyo Guardeño, tutor de este Trabajo Fin de Máster, por su gran dedicación y apoyo recibido de forma desinteresada tanto en este trabajo como durante mi paso por la Escuela Politécnica Superior. Sin duda alguna, un gran profesor y mejor persona.}
			
		\vspace{1cm}	
		
		\textit{Mención también a la Escuela Politécnica Superior por la ayuda proporcionada para la obtención del hardware empleado en el Trabajo Fin de Máster.}
	\end{flushright}
	
	\blankpage
	
	% Abstract
	\chapter*{Resumen}
	\addcontentsline{toc}{chapter}{Resumen}
	\thispagestyle{empty}
	
	\paragraph{Palabras clave \\}
		
	Cadena de bloques, Libro principal, Contratos, Internet de las Cosas.
			
	\blankpage
	
	% Abstract
	\chapter*{Abstract}
	\addcontentsline{toc}{chapter}{Abstract}
	\thispagestyle{empty}
	
	\paragraph{Keywords \\}
		
	Blockchain, Ledger, Contracts, Internet of Things.
 	
	\blankpage
	
	% Index
	\tableofcontents
	\addcontentsline{toc}{chapter}{Índice general}	
	\thispagestyle{empty}
	
	\listoffigures
	\addcontentsline{toc}{chapter}{Índice de figuras}
	\thispagestyle{empty}
		
	\pagestyle{empty}
	\renewcommand{\tablename}{Tabla}
	\renewcommand\listtablename{Índice de tablas}
	\listoftables
	\addcontentsline{toc}{chapter}{\listtablename}
	\thispagestyle{empty}
	
	% Glossary of terms
	\chapter*{Glosario de términos} \label{glosario}
	\addcontentsline{toc}{chapter}{Glosario de términos}
	\thispagestyle{empty}

	\noindent
	\textbf{Application Programming Interface (API):} Interfaz de programación de aplicaciones; conjunto de subrutinas, funciones y procedimientos que ofrece cierta biblioteca para ser utilizado por otro software como una capa de abstracción. \\
	\textbf{Autoridad certificadora (CA):} Entidad de confianza responsable de emitir y revocar los certificados digitales o certificados utilizados en la firma electrónica, para lo cual se emplea la criptografía de clave pública. \\
	\textbf{Back-end:} Parte del \textit{software} que procesa la entrada desde el \textit{front-end}, es decir, se encarga de la lógica y acceso a la base de datos del sistema. \\
	\textbf{CRUD:} Acrónimo utilizado para referirse a las funciones básicas en bases de datos o la capa de persistencia en un \textit{software}. \\
	\textbf{Diagrama de Gantt:} Herramienta gráfica cuyo objetivo es exponer el tiempo de dedicación previsto para diferentes tareas o actividades a lo largo de un tiempo total determinado. \\
	\textbf{Framework:} Esquema estandarizado de conceptos, prácticas y criterios que se postulan como base para el desarrollo y/o la implementación de una aplicación. \\
	\textbf{Front-end:} Parte del \textit{software} con la cual interactúa con el usuario final, es decir, la interfaz web. \\ 
	\textbf{Full stack:} \textit{Framework} que proporciona componentes para todas las capas de la arquitectura de desarrollo \textit{software}, es decir, se puede engloba tanto la parte \textit{front-end} como el \textit{back-end}. \\
	\textbf{Función resumen (Hash):} Algoritmo para crear a partir de una entrada, una salida alfanumérica de longitud fija que representa un resumen de toda la información que se le ha proporcionado. \\
	\textbf{HyperText Markup Language (HTML):} Lenguaje de marcas de hipertexto para la elaboración de páginas web. \\
	\textbf{Hypertext Transfer Protocol (HTTP):} Protocolo de transferencia de hipertexto; Protocolo sin estado de comunicación que permite la transferencia de información en la \textit{World Wide Web}. Define la sintaxis y semántica que utilizan los elementos de \textit{software} de la arquitectura web (clientes, servidores) para comunicarse. \\
	\textbf{Hypertext Transfer Protocol Secure (HTTPS):} Protocolo de aplicación basado en el protocolo HTTP, destinado a la transferencia segura de datos. Proporciona una comunicación segura entre origen y destino. \\
	\textbf{Paradigma cliente-servidor:} Patrón arquitectónico para el desarrollo de sistemas distribuidos basados en el concepto de diálogo petición-respuesta. Distribuye la aplicación en 2 tipos de entidades con rol bien diferenciado: cliente y servidor. \\
	\textbf{Patrón de diseño:} Base para la búsqueda de soluciones a problemas comunes en el desarrollo de \textit{software}. \\
	\textbf{Plugin:} Complemento para agregar una función nueva a la aplicación. \\
	\textbf{Programación orientada a objetos (POO):} Paradigma de programación que usa objetos en sus interacciones para diseñar aplicaciones informáticas. Se basa en técnicas como: herencia, abstracción, polimorfismo, encapsulamiento, etc. \\
	\textbf{Representational State Transfer (REST):} Transferencia de estado representacional; conjunto de técnicas orientadas a crear servicios web en los que se renuncia a la posibilidad de especificar la interfaz de los servicios de forma abstracta a cambio de contar con una convención que permite manejar la información mediante una serie de operaciones estándar. \\
	\textbf{Stakeholder:} En gestión de proyectos, todas aquellas personas u organizaciones que afectan o son afectadas por el proyecto, ya sea de forma positiva o negativa. \\
	\textbf{Structured Query Language (SQL):} Lenguaje de consulta estructurado; lenguaje declarativo de acceso a bases de datos relacionales que permite especificar diversos tipos de operaciones en ellas. \\
	\textbf{Uniform Resource Identifier (URI):} Identificador de recursos uniforme; cadena de caracteres que identifica los recursos de una red de forma unívoca. \\
	\textbf{Uniform Resource Locator (URL):} Localizador de recursos uniforme; identificador de recursos uniforme cuyos recursos referidos pueden cambiar. \\
	\textbf{XX:} \\
	\textbf{XX:} \\
	\textbf{XX:} \\
	\textbf{XX:} \\
	\textbf{XX:} \\
	\textbf{XX:} \\
	\textbf{XX:} \\
	\textbf{XX:} \\
	\textbf{XX:} \\
	\textbf{XX:} \\
	\textbf{XX:} \\
	\textbf{XX:} \\
	\textbf{XX:} \\
	\textbf{XX:} \\
	\textbf{XX:} \\
	\textbf{XX:} \\
	\textbf{XX:} \\
	\textbf{XX:} \\
		 	
	\thispagestyle{empty}
	
	% Header
	\pagestyle{fancy}
	% Right side
	\fancyhead[L]{}
	% Center side
	\fancyhead[C]{\small{\textit{\titleTFM}}}
	% Right side
	\fancyhead[R]{}
	
	% Header width
	\setlength{\headwidth}{17.5cm}
	% Header line thickness
	\renewcommand{\headrulewidth}{1.3pt}
	%\renewcommand{\footrulewidth}{0.8pt}
	%\setlength{\topmargin}{0.1cm}
	\setlength{\headsep}{1.1cm}
	%\oddsidemargin = 10pt
	
	% Introduction chapter
	\chapter{Introducción y objetivos} \label{introductionChapter}
	
	% Start arabic numerals
	\pagenumbering{arabic}
	
	\textbf{Resumen:} \textit{XXX}
	
	% Section
	\section{Definición del problema}
	
	% Section
	\section{Justificación}
	
	% Section
	\section{Objetivo general}
	
	% Section
	\section{Herramientas empleadas}
	
	% Section
	\section{Estructura del documento}
	
	% State of the art chapter
	\chapter{Estado del arte} \label{stateArtChapter}
	\textbf{Resumen:} \textit{XXX}
	
	% Planning chapter
	\chapter{Planificación del proyecto} \label{planningChapter}
	\textbf{Resumen:} \textit{XXX}
	
	% Section
	\section{Proceso de desarrollo de software}
	
	% Section
	\section{Organización del proyecto. Estructura interna}
	
	% Section
	\section{Plan de fases}
	
	% Section
	\section{Plan de iteraciones}
	
	% Section
	\section{Estimación de costes}
	
	% Section
	\section{Gestión de riesgos}
	
	% Subsection
	\subsection{Identificación y análisis de riesgos}
		
	% Analysis chapter
	\chapter{Análisis} \label{analysisChapter}
	
	\textbf{Resumen:} \textit{XXX}
	
	% Design chapter
	\chapter{Diseño} \label{designChapter}
	
	\textbf{Resumen:} \textit{XXX}
	
	% Implementation chapter
	\chapter{Implementación} \label{implementationChapter}
	
	\textbf{Resumen:} \textit{XXX}
	
	% Testing chapter
	\chapter{Casos de prueba} \label{testingChapter}
	
	\textbf{Resumen:} \textit{XXX}

	% Conclusions
	\chapter*{Conclusiones y trabajo futuro} \label{conclusions}
	\addcontentsline{toc}{chapter}{Conclusiones y trabajo futuro}
	
	{\Large\textbf{Conclusiones}} \\
			
	{\Large\textbf{Trabajo futuro}} \\
			
	% Bibliography
	\begin{thebibliography} {x}
	\addcontentsline{toc}{chapter}{Bibliografía}
	
		\pagestyle{empty}
		\thispagestyle{empty}

		\begingroup % Forcing linespace in url
		\raggedright 
		\sloppy
	
		\bibitem{tfm1} \textsc{\textbf{Androulaki, Elli; Cachin, Christian; De Caro, Angelo; Kind, Andreas; Osborne, Mike; Schubert, Simon; Sorniotti, Alessandro; Vukolic, Marko; }}\textsc{“Cryptography and Protocols in Hyperledger Fabric”.} Real-World Cryptography Conference 2017.
		\newline \texttt{\textbf{Fecha última consulta:} XX de XX de 2017}.
		
		\bibitem{tfm2} \textsc{\textbf{Bahga, Arshdeep; Masidetti, Vijay; }}\textsc{“Blockchain Platform for Industrial Internet of Things”.} 2016.
		\newline \texttt{\textbf{Fecha última consulta:} XX de XX de 2017}.
	
		\bibitem{tfm3} \textsc{\textbf{Bhowmik, Deepayan; Feng, Tian; }}\textsc{“The Multimedia Blockchain: A Distributed and Tamper - Proof Media Transaction Framework”.} 2017.
		\newline \texttt{\textbf{Fecha última consulta:} XX de XX de 2017}.

		\bibitem{tfm4} \textsc{\textbf{Cachin, Christian (IBM Research); }}\textsc{“Architecture of the Hyperledger Blockchain Fabric”.} 2016. 
		\newline \texttt{\textbf{Fecha última consulta:} XX de XX de 2017}.
		
		\bibitem{tfm5} \textsc{\textbf{Christidis, Konstantinos; Devetsikiotis, Michael; }}\textsc{“Blockchains and Smart Contracts for the Internet of Things”.} 2016.
		\newline \texttt{\textbf{Fecha última consulta:} XX de XX de 2017}.

		\bibitem{tfm6} \textsc{\textbf{Conoscenti, Marco; Vetrò, Antonio; De Martin, Juan Carlos; }}\textsc{“Blockchain for the Internet of Things: a Systematic Literature Review”.} 2016.
		\newline \texttt{\textbf{Fecha última consulta:} XX de XX de 2017}.
		
		\bibitem{tfm7} \textsc{\textbf{Dorri, Ali; Kanhere, Salil; Jurdak, Raja; }}\textsc{“Towards an Optimized BlockChain for IoT”.} 2017.
		\newline \texttt{\textbf{Fecha última consulta:} XX de XX de 2017}.
		
		\bibitem{tfm8} \textsc{\textbf{Franco, Pedro; }}\textsc{“Understanding Bitcoin: Cryptography, Engineering and Economics. Wiley Finance Series”.} 2014. 
		\newline \texttt{\textbf{Fecha última consulta:} XX de XX de 2017}.				
		\bibitem{tfm9} \textsc{\textbf{Ghaffari, Zahra; }}\textsc{“On the Application Areas of Blockchain”.} 2016.
		\newline \texttt{\textbf{Fecha última consulta:} XX de XX de 2017}.
		
		\bibitem{tfm10} \textsc{\textbf{Harz, Dominic; }}\textsc{“Trust and Verifiable Computation for Smart Contracts in Permissionless Blockchains”.} 2017.
		\newline \texttt{\textbf{Fecha última consulta:} XX de XX de 2017}.
		
		\bibitem{tfm11} \textsc{\textbf{Hyperledger; }} \\ 
		\url{https://www.hyperledger.org/} 
		\newline \texttt{\textbf{Fecha última consulta:} XX de XX de 2017}.
		
		\bibitem{tfm12} \textsc{\textbf{Hyperledger Composer; }} \\ 
		\url{https://hyperledger.github.io/composer/} 
		\newline \texttt{\textbf{Fecha última consulta:} XX de XX de 2017}.
		
		\bibitem{tfm20} \textsc{\textbf{IBM Cloud; }} \\ 
		\url{https://www.ibm.com/cloud-computing/es/es/} 
		\newline \texttt{\textbf{Fecha última consulta:} XX de XX de 2017}.	
		
		\bibitem{tfm13} \textsc{\textbf{Narayanan, Arvind; Bonneau, Joseph; Felten, Edward; Miller, Andrew; Goldfeder, Steve; }}\textsc{“Bitcoin and Cryptoconcurrency Technologies”.} Princeton University Press, 2016.
		\newline \texttt{\textbf{Fecha última consulta:} XX de XX de 2017}.
		
		\bibitem{tfm14} \textsc{\textbf{Open Timestamps; }} \\ 
		\url{https://opentimestamps.org/} 
		\newline \texttt{\textbf{Fecha última consulta:} XX de XX de 2017}.
		
		\bibitem{tfm15} \textsc{\textbf{Peters, Gareth; Panayi, Efstathios; Chapelle, Ariane; }}\textsc{“Trends in Crypto-currencies and Blockchain Technologies: A Monetary Theory and Regulation Perspective”.} 2015.
		\newline \texttt{\textbf{Fecha última consulta:} XX de XX de 2017}.
		
		\bibitem{tfm16} \textsc{\textbf{Pilkington, Marc; }}\textsc{“Blockchain Technology: Principles and Applications”.} 2016.
		\newline \texttt{\textbf{Fecha última consulta:} XX de XX de 2017}.
		
		\bibitem{tfm17} \textsc{\textbf{Shafagh, Hossein; Hithnawi, Anwar; }}\textsc{“Towards Blockchain-based Auditable Storage and Sharing of IoT Data”.} 2017.
		\newline \texttt{\textbf{Fecha última consulta:} XX de XX de 2017}.
		
		\bibitem{tfm18} \textsc{\textbf{Vukolić, Marko; }}\textsc{“Hyperledger Fabric - An Open-Source Distributed Operating System for Permissioned Blockchains”.} Swiss Blockchain Summer School Lausann, 2017.
		\newline \texttt{\textbf{Fecha última consulta:} XX de XX de 2017}.
		
		\bibitem{tfm19} \textsc{\textbf{Zhang, Yu; Wen, Jiangtao; }}\textsc{“The IoT electric business model: Using Blockchain Technology for the Internet of Things”.} 2016.
		\newline \texttt{\textbf{Fecha última consulta:} XX de XX de 2017}.
																				
		\bibitem{XXX} \textsc{\textbf{XXX; }}
		\newline \texttt{\textbf{Fecha última consulta:} XX de XX de 2017}.
						
		\endgroup
	\end{thebibliography}
	
	\thispagestyle{empty}
		
	% Annexes
	\renewcommand{\appendixname}{Anexo}
	\renewcommand{\appendixtocname}{Anexos}
	\renewcommand{\appendixpagename}{ANEXOS} 
	\appendix 
	\addappheadtotoc 
	\appendixpage 
	
	\blankpage

	% Chapter
	\chapter{Anexo A}\label{AnnexeA}

	% Chapter
	\chapter{Anexo B}\label{AnnexeB}
	
\end{document}
	