%---------------%
% ONLY ACRONYMS %
%---------------%

% AUP
\newglossaryentry{aup-a}{type=\acronymtype, name={AUP}, description={Agile Unified Process}, see=[]{}}

% BBDD
\newglossaryentry{bbdd-a}{type=\acronymtype, name={BBDD}, description={Base de Datos}, see=[]{}}

% RUP
\newglossaryentry{rup-a}{type=\acronymtype, name={RUP}, description={Rational Unified Process}, see=[]{}}

% TFM
\newglossaryentry{tfm-a}{type=\acronymtype, name={TFM}, description={Trabajo Fin de Máster}, see=[]{}}

% TI
\newglossaryentry{ti-a}{type=\acronymtype, name={TI}, description={Tecnologías de la Información}, see=[]{}}

% TPC
\newglossaryentry{tpc-a}{type=\acronymtype, name={TPC}, description={Tercera Parte de Confianza}, see=[]{}}


%--------------------%
% GLOSSARY - ENTRIES %
%--------------------%

% AES
\newglossaryentry{aes}
{
    name={Advanced Encryption Standard (AES)},
    text={Advanced Encryption Standard},
    description={Estándar de encriptación avanzada; también conocido como Rijndael, es un esquema de cifrado por bloques adoptado como un estándar de cifrado por el gobierno de los Estados Unidos. Fue anunciado por el Instituto Nacional de Estándares y Tecnología (\textit{National Institute of Standards and Technology} -NIST-) y es considerado como uno de los algoritmos más populares usados en criptografía simétrica},
    symbol={AES},
}
\newglossaryentry{aes-a}{type=\acronymtype, name={AES}, description={Advanced Encryption Standard. \textit{Glosario:} \glslink{aes}{AES}}, see=[]{}}

% Android
\newglossaryentry{android}
{
    name={Android},
    text={Android},
    description={Sistema operativo, basado en GNU/Linux, que se emplea principalmente en dispositivos móviles, aunque actualmente su uso se ha extendido a otro tipo de dispositivos: televisores, automóviles, etc. Inicialmente fue desarrollado por Android Inc., empresa que Google respaldó económicamente y más tarde, en 2005, compró},
}

% API
\newglossaryentry{api}
{
    name={Application Programming Interface (API)},
    text={Application Programming Interface},
    description={Interfaz de programación de aplicaciones; conjunto de subrutinas, funciones y procedimientos que ofrece cierta biblioteca para ser utilizado por otro \textit{software} como una capa de abstracción},
    symbol={API},
}
\newglossaryentry{api-a}{type=\acronymtype, name={API}, description={Application
Programming Interface. \textit{Glosario:} \glslink{api}{API}}, see=[]{}}

% ASCII
\newglossaryentry{ascii}
{
    name={American Standard Code for Information Interchange (ASCII)},
    text={American Standard Code for Information Interchange},
    description={Código Estándar Estadounidense para el Intercambio de Información; código de caracteres basado en el alfabeto latino. Fue creado en 1963 por el Instituto Estadounidense de Estándares Nacionales (\textit{American National Standards Institute} -ANSI-) como una evolución de los conjuntos de códigos utilizados entonces en telegrafía},
    symbol={ASCII},
}
\newglossaryentry{ascii-a}{type=\acronymtype, name={ASCII}, description={American Standard Code for Information Interchange. \textit{Glosario:} \glslink{ascii}{ASCII}}, see=[]{}}

% Asset
\newglossaryentry{asset}
{
    name={Asset},
    text={asset},
    description={Activo; concepto relativo a Hyperledger Composer (HC) y la Blockchain (BC) de Hyperledger Fabric (HF) que hace referencia a los bienes, servicios o propiedades tanto tangibles como intangibles que son almacenados en registros en la Blockchain (BC). Dicho de otra forma, es cualquier cosa del mundo real que pueda ser representado y utilizado en una red de negocio. Este concepto es común a otros tipos de Blockchain (BC)},
}

% Bitcoin
\newglossaryentry{bitcoin}
{
    name={Bitcoin},
    text={Bitcoin},
    description={Red P2P (\textit{Peer-to-Peer}) descentralizada y distribuida desarrollada por Satoshi Nakamoto (aunque su identidad real es desconocida) en 2008 que se utiliza como sistema de pago para la transferencia de valor empleando para ello la criptomoneda digital, bitcoin. Se sustenta en la Blockchain (BC) y fue la primera aplicación de ella},
}
	
% Blockchain
\newglossaryentry{blockchain}
{
    name={Blockchain (BC)},
    text={blockchain},
    description={Cadena de bloques; libro de contabilidad (\textit{ledger}) o base de datos (BBDD) distribuida y descentralizada que es administrada y compartida entre muchas partes diferentes pertenecientes a una red de punto a punto (\textit{peer-to-peer} -P2P-). Registra bloques de información de forma permanente y garantiza la integridad de esta información},
    symbol={BC},
}
\newglossaryentry{blockchain-a}{type=\acronymtype, name={BC}, description={Blockchain. \textit{Glosario:} \glslink{blockchain}{BC}}, see=[]{}}
	
% Bourne shell
\newglossaryentry{bourneshell}
{
    name={Bourne Shell (sh)},
    text={Bourne Shell},
    description={Intérprete de comandos considerado con la \textit{shell} original de Unix. Fue desarrollada por Stephen Bourne de la compañía AT\&T y liberada en el año 1979 en la versión 7 de Unix},
    symbol={sh},
}	
\newglossaryentry{bourneshell-a}{type=\acronymtype, name={sh}, description={Bourne Shell. \textit{Glosario:} \glslink{bourneshell}{sh}}, see=[]{}}

% Bash
\newglossaryentry{bash}
{
    name={Bourne-again Shell (Bash)},
    text={Bourne-again Shell},
    description={Programa informático cuya función es interpretar por consola comandos e instrucciones de un lenguaje de programación. Está basado en la \textit{shell} de Unix y es el intérprete de comandos por defecto en la mayoría de las distribuciones de Linux},
    symbol={Bash},
}	
\newglossaryentry{bash-a}{type=\acronymtype, name={Bash}, description={Bourne-again Shell. \textit{Glosario:} \glslink{bash}{Bash}}, see=[]{}}
	
% Breadboard
\newglossaryentry{breadboard}
{
    name={Breadboard},
    text={breadboard},
    description={Placa de pruebas; dispositivo sin soldadura para prototipos temporales con diseños de circuitos electrónicos. Está compuesto de un tablero con orificios que se encuentran conectados eléctricamente entre sí de manera interna, habitualmente siguiendo patrones de líneas, en el cual se pueden insertar componentes electrónicos y cables para su conexión},
}

% Bug
\newglossaryentry{bug}
{
    name={Bug},
    text={Bug},
    description={Error de \textit{software}; problema en un programa informático que desencadena un resultado indeseado},
}

% CA
\newglossaryentry{ca}
{
    name={Certificate Authority (CA)},
    text={Certificate Authority},
    description={Autoridad de certificación; entidad de confianza responsable de emitir y revocar los certificados digitales o certificados utilizados en la firma electrónica, para lo cual se emplea criptografía de clave pública},
    symbol={CA},
}
\newglossaryentry{ca-a}{type=\acronymtype, name={CA}, description={Certificate Authority. \textit{Glosario:} \glslink{ca}{CA}}, see=[]{}}

% Certificado autofirmado
\newglossaryentry{certautofirmado}
{
    name={Certificado Autofirmado},
    text={certificado autofirmado},
    description={Fichero informático que asocia una serie de datos de identidad a una persona física, organismo o empresa confirmando de esta manera la identidad digital en Internet. Este tipo de certificado no es firmado por una autoridad certificadora (\textit{Certificate Authority} -CA-) sino que es firmado con la propia clave privada y es únicamente utilizado para desarrollo y pruebas del sistema},
}

% Certificado SSL
\newglossaryentry{certssl}
{
    name={Certificado SSL},
    text={certificado SSL},
    description={Certificado expedido por una autoridad certificadora (\textit{Certificate Authority} -CA-) que acredita la identidad y las credenciales del servidor de tal manera que se garantiza que es auténtico, real y confiable para los usuarios visitantes proporcionando por tanto seguridad a éstos},
    plural={certificados SSL},
}

% Chaincode
\newglossaryentry{chaincode}
{
    name={Chaincode},
    text={chaincode},
    description={Contrato inteligente; Programa informático, almacenado en la Blockchain (BC), el cual nadie controla y todo el mundo puede confiar que es capaz de ejecutarse y hacerse cumplir por sí mismo, de manera autónoma y automática, sin intermediarios ni mediadores como si de un contrato con cláusulas contractuales se tratara},
}

% Cloudant NoSQL DB
\newglossaryentry{cloudant}
{
    name={Cloudant NoSQL DB},
    text={Cloudant NoSQL DB},
    description={Producto \textit{software} de base de datos (BBDD) distribuida en la nube perteneciente al servicio IBM Cloud. Es de tipo no relacional (NoSQL) basada en documentos JSON (\textit{JavaScript Object Notation}) que está completamente gestionada y optimizada para la disponibilidad de datos, durabilidad y movilidad mediante una indexación avanzada. Se respalda en el gestor de bases de datos de código abierto Apache CouchDB},
}

% Cloud Foundry
\newglossaryentry{cloudfoundry}
{
    name={Cloud Foundry},
    text={Cloud Foundry},
    description={Plataforma como servicio (\textit{Platform as a Service} -PaaS-) de código abierto que apoya el ciclo de vida completo de aplicaciones de factor 12 -aplicaciones diseñadas para operar de forma adecuada en la nube cumpliendo 12 requisitos- desarrolladas en diversos lenguajes},
}

% Cookie
\newglossaryentry{cookie}
{
    name={Cookie},
    text={cookie},
    description={Galleta informática; archivo creado por un sitio web que contiene una cantidad reducida de información y que se envía entre un emisor y un receptor con el propósito de identificar al usuario almacenando su historial de actividad, de manera que se le pueda ofrecer el contenido más apropiado},
}

% Código QR
\newglossaryentry{qr}
{
    name={Código QR},
    text={código QR},
    description={Código de respuesta rápida; código de barras bidimensional cuadrada que puede almacenar información codificada},
    plural={códigos QR},
}

% Criptografía asimétrica
\newglossaryentry{criptoasim}
{
    name={Criptografía Asimétrica},
    text={criptografía asimétrica},
    description={También conocida como criptografía de clave pública o criptografía de dos claves es un método criptográfico donde se utiliza un par de claves para la comunicación. Por un lado, se encuentra la clave pública que se puede difundir sin ningún problema y por otro lado la clave privada que nunca debe ser revelada. Para enviar un mensaje, el remitente usa la clave pública del destinatario para cifrar el mensaje. Una vez que lo ha cifrado, solamente con la clave privada del destinatario se puede descifrar, ni siquiera el que ha cifrado el mensaje puede volver a descifrarlo. Por ello, se puede dar a conocer perfectamente la clave pública para que todo aquel que se quiera comunicar con el destinatario lo pueda hacer. Este tipo de criptografía es más segura que la criptografía simétrica pero requiere de mayor tiempo},
}

% Criptografía simétrica
\newglossaryentry{criptosim}
{
    name={Criptografía Simétrica},
    text={criptografía simétrica},
    description={También conocida como criptografía de clave secreta o privada o criptografía de una clave es un método criptográfico donde se utiliza la misma clave tanto para cifrar como para descifrar mensajes. La seguridad reside en la propia clave privada, y por tanto el principal problema es la distribución de claves entre el emisor y receptor ya que ambos deben usar la misma clave de forma que se tiene que buscar también un canal de comunicación que sea seguro para el intercambio de la clave ya que si una persona no autorizada se hace con dicha clave la comunicación ya no podría considerarse como segura},
}

% CSI
\newglossaryentry{csi}
{
    name={Camera Serial Interface (CSI)},
    text={Camera Serial Interface},
    description={Interfaz serie para cámaras; especificación de la MIPI Alliance que define la interfaz entre una cámara digital y un procesador anfitrión},
    symbol={CSI},
}
\newglossaryentry{csi-a}{type=\acronymtype, name={CSI}, description={Camera Serial Interface. \textit{Glosario:} \glslink{csi}{CSI}}, see=[]{}}

% CSS
\newglossaryentry{css}
{
    name={Cascading Style Sheets (CSS)},
    text={Cascading Style Sheets},
    description={Hojas de estilo en cascada; lenguaje de diseño utilizado para definir y crear la presentación de documentos estructurados y escritos con un lenguaje de marcado, como HTML (\textit{HyperText Markup Language})},
    symbol={CSS},
}
\newglossaryentry{css-a}{type=\acronymtype, name={CSS}, description={Cascading Style Sheets. \textit{Glosario:} \glslink{css}{CSS}}, see=[]{}}

% CSR
\newglossaryentry{csr}
{
    name={Certificate Signing Request (CSR)},
    text={Certificate Signing Request},
    description={Solicitud de firma de certificado; bloque de texto cifrado que contiene toda la información de la petición de certificado (nombre, dirección, dominio para el que es generado, clave pública, etc.), la cual será incluida finalmente en el certificado SSL. La autoridad certificadora (\textit{Certificate Authority} -CA-) utilizará esta solicitud para generar el certificado SSL},
    symbol={CSR},
}
\newglossaryentry{csr-a}{type=\acronymtype, name={CSR}, description={Certificate Signing Request. \textit{Glosario:} \glslink{csr}{CSR}}, see=[]{}}

% DBaaS
\newglossaryentry{dbaas}
{
    name={Database as a Service (DBaaS)},
    text={Database as a Service},
    description={Base de datos como servicio; modelo de servicio de computación en la nube que permite a los usuarios provisionar, gestionar, configurar y operar con bases de datos (BBDD) y a las aplicaciones acceder a la información, abstrayendo de la necesidad de establecer una configuración de \textit{software} o física de \textit{hardware}},
    symbol={DBaaS}
}
\newglossaryentry{dbaas-a}{type=\acronymtype, name={DBaaS}, description={Database as a Service. \textit{Glosario:} \glslink{dbaas}{DBaaS}}, see=[]{}}	

% Diagrama de Gantt
\newglossaryentry{gantt}
{
    name={Diagrama de Gantt},
    text={diagrama de Gantt},
    description={Herramienta gráfica cuyo objetivo es exponer el tiempo de dedicación previsto para diferentes tareas o actividades a lo largo de un tiempo total determinado},
    plural={diagramas de Gantt},
}

% Divisor de voltaje 
\newglossaryentry{divisor}
{
    name={Divisor de Voltaje},
    text={divisor de voltaje},
    description={Sistema de resistencias, habitualmente configurado con dos (R1 y R2), conectadas en serie donde una de ellas debe tener el doble de resistencia eléctrica que la otra. Este sistema tiene como cometido fijar la tensión a un nivel intermedio entre el nivel de alimentación del conjunto (por ejemplo, 5V) y el nivel de tierra},
}

 % DLT
 % TODO - Añadir en el capítulo para que el link del resumen funcione
\newglossaryentry{dlt}
{
    name={Distributed Ledger Technology (DLT)},
    text={Distributed Ledger Technology},
    description={Tecnología de registro distribuido; sistema digital para el almacenamiento en múltiples lugares y al mismo tiempo de transacciones de activos evitando así la necesidad de existencia de una autoridad central que almacene y administre los datos. El tipo de registro distribuido más conocido es Blockchain (BC)},
    symbol={DLT},
}
\newglossaryentry{dlt-a}{type=\acronymtype, name={DLT}, description={Distributed Ledger Technology. \textit{Glosario:} \glslink{dlt}{DLT}}, see=[]{}}

% Docker
\newglossaryentry{docker}
{
    name={Docker},
    text={Docker},
    description={Proyecto de código abierto que automatiza el despliegue de aplicaciones dentro de contenedores -vistos como máquinas virtuales (\textit{Virtual Machines} -VMs-) extremadamente livianas, modulares y portables- de \textit{software}, proporcionando una capa adicional de abstracción y automatización de virtualización a nivel de sistema operativo},
}

% ECMAScript
\newglossaryentry{ecmascript}
{
    name={ECMAScript},
    text={ECMAScript},
    description={Especificación de lenguaje de programación publicada por ECMA International y actualmente aceptado como el estándar ISO 16262. Comenzó a ser desarrollado en el año 1996, basándose en el popular lenguaje JavaScript (JS) propuesto por la compañía Netscape Communications Corporation},
}

% Entropía
\newglossaryentry{entropia}
{
    name={Entropia},
    text={entropia},
    description={Aleatoriedad recogida por un sistema operativo (SO) o una aplicación para su uso en criptografía o para otros usos que requieren datos aleatorios. Se suele obtener a partir de fuente de \textit{hardware}, tales como los movimientos del ratón o pulsaciones de teclado},
}

% ESLint
\newglossaryentry{eslint}
{
    name={ESLint},
    text={ESLint},
    description={Proyecto \textit{open source} creado originalmente por Nicholas C. Zakas en 2013 con la finalidad de ofrecer un \textit{linter} para JavaScript},
}

% Ethereum
\newglossaryentry{ethereum}
{
    name={Ethereum},
    text={Ethereum},
    description={Plataforma \textit{open source} y descentralizada propuesta por Vitalik Buterin que permite la creación de acuerdos de contratos inteligentes entre pares, basada en el modelo de Blockchain (BC). También, provee una criptomoneda llamada Ether},
}

% Ethernet
\newglossaryentry{ethernet}
{
    name={Ethernet},
    text={ethernet},
    description={Estándar, también conocido como IEEE 802.3, de redes de área local (\textit{Local Area Network} -LAN-) que determina las particularidades físicas y eléctricas que debe poseer una red tendida con este sistema},
}

% Framework
\newglossaryentry{framework}
{
    name={Framework},
    text={framework},
    description={Esquema estandarizado de conceptos, prácticas y criterios que se postulan como base para el desarrollo y/o la implementación de una aplicación},
}

% Front-End
\newglossaryentry{frontend}
{
    name={Front-End},
    text={front-end},
    description={Parte del \textit{software} que interactúa con los usuarios finales, es decir, se corresponde con la interfaz web que visualizan},
}

% Función hash
\newglossaryentry{hash}
{
    name={Función Hash},
    text={función hash},
    description={Función resumen; algoritmo para crear a partir de una entrada, una salida alfanumérica de longitud fija que representa un resumen de toda la información que se le ha proporcionado},
    plural={funciones hash},
}

% GPIO
\newglossaryentry{gpio}
{
    name={General Purpose Input Output (GPIO)},
    text={General Purpose Input Output},
    description={Entrada/salida de propósito general; pin o conexión genérica en un chip cuyo comportamiento, ya sea un pin de entrada o de salida, puede ser controlado por el usuario en tiempo de ejecución},
    symbol={GPIO},
}
\newglossaryentry{gpio-a}{type=\acronymtype, name={GPIO}, description={General Purpose Input Output. \textit{Glosario:} \glslink{gpio}{GPIO}}, see=[]{}}

% Git
\newglossaryentry{git}
{
    name={Git},
    text={Git},
    description={Sistema de control de versiones distribuido y eficiente diseñado por Linus Torvalds que permite llevar un registro de los cambios aplicados a ficheros de un proyecto y coordinar el trabajo entre desarrolladores},
}

% GNU/Linux
\newglossaryentry{gnulinux}
{
    name={GNU/Linux},
    text={GNU/Linux},
    description={Conocido como Linux, es un sistema operativo libre, multiplataforma, multiusuario y multitarea basado en Unix. Su origen se remonta a la combinación de varios proyectos, entre los que destaca GNU y el núcleo Linux, encabezado por Linus Torvalds},
}

% GPG
\newglossaryentry{gpg}
{
    name={GNU Privacy Guard (GPG)},
    text={GNU Privacy Guard},
    description={Herramienta multiplataforma de cifrado y firmas digitales desarrollado por Werner Koch, que viene a ser un reemplazo de PGP (\textit{Pretty Good Privacy}) pero con la principal diferencia que es \textit{software} libre},
    symbol={GPG},
}
\newglossaryentry{gpg-a}{type=\acronymtype, name={GPG}, description={GNU Privacy Guard. \textit{Glosario:} \glslink{gpg}{GPG}}, see=[]{}}

% Grails
\newglossaryentry{grails}
{
    name={Grails},
    text={Grails},
    description={Conocido Groovy and Rails, es un \textit{framework} de desarrollo de aplicaciones web desarrollado sobre el lenguaje de programación Groovy que potencia el desarrollo ágil},
}

% Groovy
\newglossaryentry{groovy}
{
    name={Groovy},
    text={Groovy},
    description={Lenguaje de programación e incluso de \textit{scripting} dinámico y orientado a objetos nacido en el año 2003 para potenciar Java proporcionando mayor productividad y flexibilidad gracias a todas las funcionalidad que ofrece. Al ejecutarse sobre la máquina virtual de Java (JVM), presenta el beneficio de poder acceder directamente a todas las API existentes en Java y por tanto usarse directamente en cualquier aplicación de este lenguaje},
}

% GSP
\newglossaryentry{gsp}
{
    name={Groovy Server Pages (GSP)},
    text={Groovy Server Pages},
    description={Tecnología de renderizado de la vista del lado de servidor basado en Groovy. Se considera una versión simplificada de la tecnología JSP (\textit{JavaServer Pages}) pero más poderosa e intuitiva que permite emplear una serie de etiquetas para ofrecer una forma más elegante de programación},
    symbol={GSP},
}
\newglossaryentry{gsp-a}{type=\acronymtype, name={GSP}, description={Groovy Server Pages. \textit{Glosario:} \glslink{gsp}{GSP}}, see=[]{}}

% GND
\newglossaryentry{gnd}
{
    name={Ground (GND)},
    text={Ground},
    description={Conexión a tierra o masa; terminal de referencia para todas las señales o una ruta común en un circuito eléctrico desde donde se pueden medir todos los voltajes y al que se debe aplicar un voltaje de referencia de cero voltios (0V)},
    symbol={GND},
}
\newglossaryentry{gnd-a}{type=\acronymtype, name={GND}, description={Ground. \textit{Glosario:} \glslink{gnd}{GND}}, see=[]{}}

% Hibernate
\newglossaryentry{hibernate}
{
    name={Hibernate},
    text={Hibernate},
    description={\textit{Framework} de código abierto para la plataforma Java que permite el manejo de bases de datos (BBDD) con el paradigma de programación orientada a objetos (POO) a través de una herramienta de mapeo objeto-relacional (ORM). Con ello, a nivel de la aplicación, las tablas se transforman en objetos, los registros se convierten en atributos del objeto, las claves foráneas se transforman en asociaciones entre objetos y las consultas se traducen en llamadas a métodos. Además, proporciona su propio lenguaje de consultas denominado HQL (\textit{Hibernate Query Language})},
}

% Historian
\newglossaryentry{historian}
{
    name={Historian},
    text={historian},
    description={Concepto relativo a Hyperledger Composer (HC) y la Blockchain (BC) de Hyperledger Fabric (HF) que hace referencia al registro especializado que almacena información sobre todas las transacciones completadas exitosamente, incluyendo los participantes (\textit{participants}) e identidades (\textit{identities}) que las subieron},
}

% HTML
\newglossaryentry{html}
{
    name={HyperText Markup Language (HTML)},
    text={HyperText Markup Language},
    description={Lenguaje de marcado de hipertexto para el desarrollo y representación visual de páginas web},
    symbol={HTML},
}
\newglossaryentry{html-a}{type=\acronymtype, name={HTML}, description={HyperText Markup Language. \textit{Glosario:} \glslink{html}{HTML}}, see=[]{}}
		
% HTTP
\newglossaryentry{http}
{
    name={Hypertext Transfer Protocol (HTTP)},
    text={Hypertext Transfer Protocol},
    description={Protocolo de transferencia de hipertexto; protocolo sin estado de comunicación que permite la transferencia de información en la \textit{World Wide Web}. Define la sintaxis y semántica que utilizan los elementos de \textit{software} de la arquitectura web (clientes, servidores) para comunicarse},
    symbol={HTTP},
}
\newglossaryentry{http-a}{type=\acronymtype, name={HTTP}, description={Hypertext Transfer Protocol. \textit{Glosario:} \glslink{http}{HTTP}}, see=[]{}}
			
% HTTPS
\newglossaryentry{https}
{
    name={Hypertext Transfer Protocol Secure (HTTPS)},
    text={Hypertext Transfer Protocol Secure},
    description={Protocolo seguro de transferencia de hipertexto; protocolo de aplicación basado en el protocolo HTTP (\textit{Hypertext Transfer Protocol}), destinado a la transferencia segura de datos. Proporciona una comunicación segura entre origen y destino},
    symbol={HTTPS},
}
\newglossaryentry{https-a}{type=\acronymtype, name={HTTPS}, description={Hypertext Transfer Protocol Secure. \textit{Glosario:} \glslink{https}{HTTPS}}, see=[]{}}
		
% Hyperledger
\newglossaryentry{hyperledger}
{
    name={Hyperledger},
    text={Hyperledger},
    description={Iniciativa de carácter colaborativo anunciado en el año 2015 por la fundación Linux para investigar y evolucionar la tecnología Blockchain (BC) de uso privado y orientada al ámbito empresarial},
}

% Hyperledger Composer
\newglossaryentry{hyperledgercomposer}
{
    name={Hyperledger Composer (HC)},
    text={Hyperledger Composer},
    description={Herramienta de código abierto, perteneciente a la iniciativa Hyperledger, para desarrollar redes de negocio (\textit{Business Network Definition} -BND-) de Blockchain (BC) de soluciones construidas sobre la infraestructura de Hyperledger Fabric, abstrayendo así la complejidad del desarrollo directo en esta última tecnología},
    symbol={HC},
}
\newglossaryentry{hyperledgercomposer-a}{type=\acronymtype, name={HC}, description={Hyperledger Composer. \textit{Glosario:} \glslink{hyperledgercomposer}{HC}}, see=[]{}}

% Hyperledger Fabric
\newglossaryentry{hyperledgerfabric}
{
    name={Hyperledger Fabric (HF)},
    text={Hyperledger Fabric},
    description={Proyecto ubicado en la iniciativa Hyperledger que implementa la tecnología de libro de registros distribuido (\textit{Distributed Ledger Technology} -DLT-) en una Blockchain (BC) de tipo permisionada. Ofrece características (arquitectura modular y escalable, red transaccional de alto rendimiento permisionada, privacidad e identidad, contratos inteligentes, etc.) que tienden a mejorar aspectos de productividad y fiabilidad distinguiéndola de otras alternativas},
    symbol={HF},
}
\newglossaryentry{hyperledgerfabric-a}{type=\acronymtype, name={HF}, description={Hyperledger Fabric. \textit{Glosario:} \glslink{hyperledgerfabric}{HF}}, see=[]{}}

% I2C
\newglossaryentry{i2c}
{
    name={Inter-Integrated Circuit (I2C)},
    text={Inter-Integrated Circuit},
    description={Circuito Interintegrado; protocolo síncrono de comunicaciones en serie que emplea dos líneas para transmitir la información: una para los datos (\textit{Serial Data} -SDA-) y otra para la señal de reloj (\textit{Serial Clock} -SCL-). También es necesaria una tercera línea, pero esta sólo es la referencia (masa). Es muy usado en la industria, principalmente para comunicar microcontroladores y sus periféricos en sistemas integrados},
    symbol={I2C},
}
\newglossaryentry{i2c-a}{type=\acronymtype, name={I2C}, description={Inter-Integrated Circuit. \textit{Glosario:} \glslink{i2c}{I2C}}, see=[]{}}

% IDE
\newglossaryentry{ide}
{
    name={Integrated Development Environment (IDE)},
    text={Integrated Development Environment},
    description={Entorno de desarrollo integrado; aplicación informática que proporciona servicios para facilitar al programador el desarrollo de \textit{software}},
    symbol={IDE},
}
\newglossaryentry{ide-a}{type=\acronymtype, name={IDE}, description={Integrated Development Environment. \textit{Glosario:} \glslink{ide}{IDE}}, see=[]{}}

% Identity
\newglossaryentry{identity}
{
    name={Identity},
    text={identity},
    description={Identidad; concepto relativo a Hyperledger Composer (HC) y la Blockchain (BC) de Hyperledger Fabric (HF) que hace referencia a la identidad que ejecuta transacciones (\textit{transactions}) en una red de negocio. Esta identidad está compuesta de un certificado digital y de una clave privada},
    plural={identities},
}
	
% Intensidad
\newglossaryentry{intensidad}
{
    name={Intensidad de Corriente Eléctrica},
    text={intensidad de corriente eléctrica},
    description={Caudal de flujo de carga eléctrica (normalmente electrones) por unidad de tiempo que recorre un material. La unidad de resistencia en el Sistema Internacional (SI) es el amperio (A)},
}

% IOS
\newglossaryentry{ios}
{
    name={iOS},
    text={iOS},
    description={Sistema operativo creado por la compañía Apple Inc. para sus dispositivos móviles: Iphone, Ipad y Ipod},
    symbol={iOS},
} 
  	  
% IOT
\newglossaryentry{iot}
{
    name={Internet of Things (IoT)},
    text={Internet of Things},
    description={Internet de las Cosas; concepto referido a la interconexión digital de objetos del mundo real con Internet, convirtiéndose así en objetos inteligentes de forma que adquieren la capacidad de transferir datos a través de la red, sin requerir de interacción humano a humano o humano a computador},
    symbol={IoT},
}
\newglossaryentry{iot-a}{type=\acronymtype, name={IoT}, description={Internet of Things. \textit{Glosario:} \glslink{iot}{IoT}}, see=[]{}}
  
% ISO/IEC 27005:2008
\newglossaryentry{iso27005}
{
    name={ISO/IEC 27005:2008},
    description={Norma estándar de la Organización Internacional de Normalización (\textit{International Organization for Standardization} -ISO-) y la Comisión Electrotécnica Internacional (\textit{International Electrotechnical Commission} -IEC-) que proporciona directrices para el proceso de gestión de riesgos de seguridad de la información y sus actividades},
}

% Java
\newglossaryentry{java}
{
    name={Java},
    text={Java},
    description={Lenguaje de programación de propósito general, concurrente, orientado a objetos, que fue diseñado por la empresa Sun Microsystem en 1991 específicamente para tener tan pocas dependencias de implementación como fuera posible con la intención de permitir que una misma aplicación pudiese ejecutarse en cualquier máquina virtual Java (JVM) sin importar la arquitectura del dispositivo},
}

% JAR
\newglossaryentry{jar}
{
    name={Java ARchive (JAR)},
    text={Java ARchive},
    description={Archivo Java; archivo que permite ejecutar aplicaciones escritas en el lenguaje Java},
	symbol={JAR},
}
\newglossaryentry{jar-a}{type=\acronymtype, name={JAR}, description={Java ARchive. \textit{Glosario:} \glslink{jar}{JAR}}, see=[]{}}

% JavaScript
\newglossaryentry{js}
{
    name={JavaScript (JS)},
    text={JavaScript},
    description={Lenguaje de programación ligero, interpretado y dialecto del estándar ECMAScript que fue inventado por Brendan Eich y se utiliza principalmente en el lado del cliente (\textit{front-end}) de sitios web. Entre otras características, se encuentran: orientación a objetos e imperativa, débilmente tipado y dinámico y basado en prototipos},
	symbol={JS},
}
\newglossaryentry{js-a}{type=\acronymtype, name={JS}, description={JavaScript. \textit{Glosario:} \glslink{js}{JS}}, see=[]{}}

% JSDoc
\newglossaryentry{jsdoc}
{
    name={JavaScript Doc (JSDoc)},
    text={JavaScript Doc},
    description={Lenguaje de marcado utilizado para documentar el código fuente de ficheros JavaScript (JS)},
    symbol={JSDoc},
}
\newglossaryentry{jsdoc-a}{type=\acronymtype, name={JSDoc}, description={JavaScript Doc. \textit{Glosario:} \glslink{jsdoc}{JSDoc}}, see=[]{}}

% JSON
\newglossaryentry{json}
{
    name={JavaScript Object Notation (JSON)},
    text={JavaScript Object Notation},
    description={Formato de texto ligero de intercambio de datos, legible para los seres humanos y fácil de interpretar y generar para las máquinas. Este formato está constituido por dos estructuras universales: una colección de pares de nombre/valor y una lista ordenada de valores},
    symbol={JSON},
}
\newglossaryentry{json-a}{type=\acronymtype, name={JSON}, description={JavaScript Object Notation. \textit{Glosario:} \glslink{json}{JSON}}, see=[]{}}

% Kanban
\newglossaryentry{kanban}
{
    name={Kanban},
    text={Kanban},
    description={Metodología formulada por David J. Anderson para el desarrollo de proyectos que se centra en mejorar la visibilidad del flujo de trabajo utilizando el tablero Kanban el cual facilita la limitación del trabajo en curso con el fin de no sobrecargar a los miembros del equipo y cumplir con los plazos de entrega},
}

% LED
\newglossaryentry{led}
{
    name={Light-Emitting Diode (LED)},
    text={Light-Emitting Diode},
    description={Diodo emisor de luz; fuente de luz constituida por un material semiconductor dotado de dos terminales},
    symbol={LED},
}
\newglossaryentry{led-a}{type=\acronymtype, name={LED}, description={Light-Emitting Diode. \textit{Glosario:} \glslink{led}{LED}}, see=[]{}}

% Lenguaje de marcado
\newglossaryentry{lenguajemarcado}
{
    name={Lenguaje de Marcado},
    text={lenguaje de marcado},
    description={También denominado como lenguaje de marcas, es un lenguaje para el procesamiento, definición y presentación de texto con el fin de que los sistemas informáticos puedan manipularlo de forma sencilla. Junto al texto, se incorporan también etiquetas que contienen información adicional acerca de la estructura del documento. El lenguaje de marcas más extendido es el HTML (\textit{HyperText Markup Language})},
}

% Ley de Ohm
\newglossaryentry{ohm}
{
    name={Ley de Ohm},
    text={ley de Ohm},
    description={Ley básica de los circuitos eléctricos, postulada por el físico y matemático alemán Georg Simon Ohm, que establece que la diferencia de potencial (V) que aplicamos entre los extremos de un conductor determinado es proporcional a la intensidad de la corriente (I) que circula por el citado conductor. Ohm completó la ley introduciendo la noción de resistencia eléctrica (R), originando la fórmula general de la ley de Ohm: V = R $\cdot$ I},
}

% LCD
\newglossaryentry{lcd}
{
    name={Liquid Crystal Display (LCD)},
    text={Liquid Crystal Display},
    description={Pantalla de cristal líquido; pantalla delgada y plana formada por un número de píxeles en color o monocromos colocados delante de una fuente de luz o reflectora},
    symbol={LCD},
    plural={LCDs},
}
\newglossaryentry{lcd-a}{type=\acronymtype, name={LCD}, description={Liquid Crystal Display. \textit{Glosario:} \glslink{lcd}{LCD}}, see=[]{}}

% Linter
\newglossaryentry{linter}
{
    name={Linter},
    text={linter},
    description={Término que designa todas aquellas herramientas que realizan tareas de comprobación y análisis estático del código fuente con el fin de proporcionar información que posibilite la mejora de la calidad de éste},
}

% MacOS
\newglossaryentry{macos}
{
    name={Macintosh Operating System (MacOS)},
    text={Macintosh Operating System},
    description={Sistema operativo creado por la compañía Apple para sus equipos de sobremesa y portátiles. Está basado en Unix y usa HFS+ (\textit{Hierarchical File System +}) para integrar un sistema de archivos propio},
    symbol={MacOS},
}
\newglossaryentry{macos-a}{type=\acronymtype, name={MacOS}, description={Macintosh Operating System. \textit{Glosario:} \glslink{macos}{MacOS}}, see=[]{}}

% Markdown
\newglossaryentry{markdown}
{
    name={Markdown},
    text={markdown},
    description={Lenguaje de marcado ligero creado por John Gruber que facilita la aplicación de formato y estilo a un texto empleando de forma especial una serie de caracteres},
}

% MVP
\newglossaryentry{mvp}
{
    name={Minimum Viable Product (MVP)},
    text={Minimum Viable Product},
    description={Producto que posee las suficientes características para satisfacer a los clientes iniciales y proporcionar retroalimentación para el desarrollo futuro y por tanto validar por lo menos una parte del negocio},
    symbol={MVP},
}
\newglossaryentry{mvp-a}{type=\acronymtype, name={MVP}, description={Minimum Viable Product. \textit{Glosario:} \glslink{mvp}{MVP}}, see=[]{}}

% Node.js
\newglossaryentry{nodejs}
{
    name={Node.js},
    text={Node.js},
    description={Entorno de ejecución \textit{open source}, multiplataforma, asíncrono y basado en eventos del lado del servidor para JavaScript (JS) que fue construido utilizando el motor V8, desarrollado por Google para uso de su navegador Google Chrome},
}

% NPM
\newglossaryentry{npm}
{
    name={Node Package Manager (NPM)},
    text={Node Package Manager},
    description={Gestor de paquetes para JavaScript (JS), utilizado también por defecto para el entorno de ejecución Node.js},
    symbol={NPM},
}
\newglossaryentry{npm-a}{type=\acronymtype, name={NPM}, description={Node Package Manager. \textit{Glosario:} \glslink{npm}{NPM}}, see=[]{}}

% Open Source
\newglossaryentry{opensource}
{
    name={Open Source},
    text={open source},
    description={De código abierto; programa cuyo código fuente se encuentra disponible para el uso y/o modificación libre por parte de otros usuarios. Este tipo de \textit{software} suele ser desarrollado por la colaboración pública y desinteresada de desarrolladores},
}

% Participant
\newglossaryentry{participant}
{
    name={Participant},
    text={participant},
    description={Participante; concepto relativo a Hyperledger Composer (HC) y la Blockchain (BC) de Hyperledger Fabric (HF) que hace referencia a los miembros de una red de negocio los cuales poseen activos (\textit{assets}) y envían transacciones (\textit{transactions}). Este concepto es común a otros tipos de Blockchain (BC)},
}

% PAAS
\newglossaryentry{paas}
{
    name={Platform as a Service (PaaS)},
    text={Platform as a Service},
    description={Plataforma como servicio; categoría de servicios en la nube que proporcionan una plataforma y un entorno que incluye todos los recursos \textit{software} necesarios para soportar el ciclo de vida completo del desarrollo y puesta en marcha de aplicaciones (diseño, desarrollo, compilación, pruebas, distribución y administración, hospedaje, etc.), donde los usuarios pueden acceder a ellas simplemente a través de un navegador web},
    symbol={PaaS}
}
\newglossaryentry{paas-a}{type=\acronymtype, name={PaaS}, description={Platform as a Service. \textit{Glosario:} \glslink{paas}{PaaS}}, see=[]{}}	

% Paradigma cliente-servidor MVC
\newglossaryentry{clienteservidor}
{
    name={Paradigma Cliente-Servidor},
    text={paradigma cliente-servidor},
    description={Patrón arquitectónico para el desarrollo de sistemas distribuidos basados en el concepto de diálogo petición-respuesta. Distribuye la aplicación en 2 tipos de entidades cada una de ellas presenta un rol bien diferenciado: cliente o servidor},
}	

% Patrón de diseño
\newglossaryentry{patrondiseño}
{
    name={Patrón de Diseño},
    text={patrón de diseño},
    description={Esqueleto de soluciones a problemas comunes en el desarrollo de \textit{software}. Los patrones de diseño se clasifican en: patrones creacionales, patrones arquitectónicos o estructurales y patrones de comportamiento},
}	
	
% Patrón MVC
\newglossaryentry{mvc}
{
    name={Patrón Modelo-Vista-Controlador (MVC)},
    text={Modelo-Vista-Controlador},
    description={Estilo de arquitectura de \textit{software} maduro e inventado en el contexto de Smalltak -lenguaje reflexivo de programación orientado a objetos- que, utilizando 3 componentes bien diferenciados, separa en una aplicación los datos de la lógica de la aplicación y de la lógica de la vista},
    symbol={MVC},
}
\newglossaryentry{mvc-a}{type=\acronymtype, name={MVC}, description={Modelo-Vista-Controlador. \textit{Glosario:} \glslink{mvc}{MVC}}, see=[]{}}

% P2P
\newglossaryentry{p2p}
{
    name={Peer-to-Peer (P2P)},
    text={peer-to-peer},
    description={Red entre pares; tipo de arquitectura para la comunicación entre aplicaciones que permite a individuos comunicarse y compartir información con otros individuos sin necesidad de una autoridad central que facilite la comunicación. Este tipo de redes optimizan el uso de recursos},
    symbol={P2P},
}
\newglossaryentry{p2p-a}{type=\acronymtype, name={P2P}, description={Peer-to-peer. \textit{Glosario:} \glslink{p2p}{P2P}}, see=[]{}}

% Plugin
\newglossaryentry{plugin}
{
    name={Plugin},
    text={plugin},
    description={Complemento \textit{software} que contiene un grupo de funciones o características específicas para agregarselas a una aplicación},
}

% Perl
\newglossaryentry{perl}
{
    name={Practical Extraction and Report Language (Perl)},
    text={Perl},
    description={Lenguaje de programación de propósito general e interpretado diseñado por Larry Wall en el año 1987 con el objetivo principal de simplificar las tareas de administración de un sistema Unix. Hereda características y estructuras de otros lenguajes, como por ejemplo del lenguae C, del lenguaje interpretado Bourne Shell (sh), etc},
}

% PoC
\newglossaryentry{poc}
{
    name={Proof of Concept (PoC)},
    text={Proof of Concept},
    description={Prueba de concepto; implementación, a menudo resumida o incompleta, de un método o idea de producto en detalle realizada con el propósito de verificar que el concepto es susceptible de ser explotado de una manera útil. El objetivo principal es valorar el concepto de producto antes de comenzar su desarrollo a nivel técnico o físico lo que constituirá el producto final},
    symbol={PoC},
}
\newglossaryentry{poc-a}{type=\acronymtype, name={PoC}, description={Proof of Concept. \textit{Glosario:} \glslink{poc}{PoC}}, see=[]{}}	
	
% Prototipo
\newglossaryentry{prototipo}
{
    name={Prototipo},
    text={prototipo},
    description={Representación inicial y limitada de un producto o sistema que posee las características de la versión final o parte de ellas y permite a las partes interesadas modelar el comportamiento en situaciones reales y explorar su uso, creando así un proceso de diseño de iteración que genera calidad},
}

% Python
\newglossaryentry{python}
{
    name={Python},
    text={Python},
    description={Lenguaje de programación creado por Guido van Rossum. Se trata de un lenguaje \textit{open source} de \textit{scripting} independiente de la plataforma e interpretado cuya filosofía hace hincapié en una sintaxis que favorezca un código legible y simple de implementar},
}

% Query
\newglossaryentry{query}
{
    name={Query},
    text={query},
    description={Concepto relativo a Hyperledger Composer (HC) y la Blockchain (BC) de Hyperledger Fabric (HF) que hace referencia al método para retornar datos de la Blockchain (BC), en concreto, la información almacenada en el \textit{World State} o estado de la base de datos (BBDD). Este concepto es común para hacer referencia a la consulta realizada contra una BBDD},
    plural={queries},
}

% RAM
\newglossaryentry{ram}
{
    name={Random Access Memory (RAM)},
    text={Random Access Memory},
    description={Memoria de acceso aleatorio; memoria principal de un dispositivo donde se almacenan datos y programas. Esta memoria es de tipo volátil lo que significa que los datos no se guardan de manera permanente, es por ello, que cuando deja de existir una fuente de energía en el dispositivo la información se pierde},
    symbol={RAM}
}
\newglossaryentry{ram-a}{type=\acronymtype, name={RAM}, description={Random Access Memory. \textit{Glosario:} \glslink{ram}{RAM}}, see=[]{}}	

% Raspberry Pi
\newglossaryentry{raspberry}
{
    name={Raspberry Pi (RPi)},
    text={Raspberry Pi},
    description={Computador de placa reducida (\textit{Single Board Computer} -SBC-) de bajo coste desarrollado en el Reino Unido por la Fundación Raspberry PI en 2011, con la misión de estimular y fomentar la enseñanza de las ciencias de la computación en las escuelas, aunque no empezó su comercialización hasta el año 2012},
    symbol={RPi}
}
\newglossaryentry{raspberry-a}{type=\acronymtype, name={RPi}, description={Raspberry Pi. \textit{Glosario:} \glslink{raspberry}{RPi}}, see=[]{}}

% Resistencia
\newglossaryentry{resistencia}
{
    name={Resistencia},
    text={resistencia},
    description={Componente electrónico diseñado para introducir una resistencia eléctrica (oposición al flujo de electrones al moverse a través de un conductor) determinada entre dos puntos de un circuito eléctrico},
}

% Resistencia eléctrica
\newglossaryentry{resistenciaElectrica}
{
    name={Resistencia Eléctrica},
    text={resistencia eléctrica},
    description={Oposición al flujo de electrones al moverse a través de un material conductor. La unidad de resistencia en el Sistema Internacional (SI) es el ohmio ($\Omega$)},
}

% REST
\newglossaryentry{rest}
{
    name={Representational State Transfer (REST)},
    text={Representational State Transfer},
    description={Transferencia de estado representacional; conjunto de técnicas orientadas a crear servicios web en los que se renuncia a la posibilidad de especificar la interfaz de los servicios de forma abstracta a cambio de contar con una convención que permite manejar la información mediante una serie de operaciones estándar},
    symbol={REST},
}
\newglossaryentry{rest-a}{type=\acronymtype, name={REST}, description={Representational State Transfer. \textit{Glosario:} \glslink{rest}{REST}}, see=[]{}}

% RSA
\newglossaryentry{rsa}
{
    name={Rivest, Shamir y Adleman (RSA)},
    text={Rivest, Shamir y Adleman},
    description={Sistema criptográfico asimétrico o de clave pública desarrollado en 1977. Es el primer algoritmo de este tipo y el más utilizado siendo válido tanto para cifrar como para firmar digitalmente. La seguridad de este algoritmo radica en el problema de la factorización de números enteros grandes con sus números primos},
    symbol={RSA},
}
\newglossaryentry{rsa-a}{type=\acronymtype, name={RSA}, description={Rivest, Shamir y Adleman. \textit{Glosario:} \glslink{rsa}{RSA}}, see=[]{}}

% Scaffolding
\newglossaryentry{scaffolding}
{
    name={Scaffolding},
    text={scaffolding},
    description={Técnica para la generación automática de código a partir de una plantilla preestablecida},
}

% Script
\newglossaryentry{script}
{
    name={Script},
    text={script},
    description={Programa, usualmente simple y de tamaño pequeño, que permite realizar tareas específicas a partir de un conjunto de órdenes definidas. Este tipo de programa no es compilado, sino que es interpretado línea a línea en tiempo real durante su ejecución para codificar la información y traducirla a lenguaje máquina. Su principal utilidad es interactuar con el sistema operativo de manera automatizada, aunque muchas veces se utilizan lenguajes interpretados, como Perl o Python, para realizar tareas más complejas},
}

% SHA
\newglossaryentry{sha}
{
    name={Secure Hash Algorithm (SHA)},
    text={Secure Hash Algorithm},
    description={Algoritmo de Hash Seguro; familia de funciones \textit{hash} de cifrado 
    publicadas por el Instituto Nacional de Normas y Tecnología (\textit{National Institute of Standards and Technology} -NIST-)​ de EE.UU. Existen varias versiones, desde la primera versión SHA-0 creada en 1993 hasta la más reciente SHA-3 publicada en 2012. Esta  última versión se caracteriza por ser la que más difiere de sus predecesoras siendo el descubrimiento de vulnerabilidades la razón de la existencia de varias versiones},
    symbol={SHA},
}
\newglossaryentry{sha-a}{type=\acronymtype, name={SHA}, description={Secure Hash Algorithm. \textit{Glosario:} \glslink{sha}{SHA}}, see=[]{}}

% SSH
\newglossaryentry{ssh}
{
    name={Secure Shell (SSH)},
    text={Secure Shell},
    description={Protocolo que facilita las comunicaciones seguras entre dos sistemas usando una arquitectura cliente-servidor y que permite a los usuarios conectarse de forma encriptada a un host remoto},
    symbol={SSH},
}
\newglossaryentry{ssh-a}{type=\acronymtype, name={SSH}, description={Secure Shell. \textit{Glosario:} \glslink{ssh}{SSH}}, see=[]{}}

% SSL
\newglossaryentry{ssl}
{
    name={Secure Sockets Layer (SSL)},
    text={Secure Sockets Layer},
    description={Capa de puertos seguros; protocolo criptográfico que proporciona autenticación y privacidad de la información entre extremos sobre Internet mediante el uso de criptografía. Habitualmente, solo el servidor es autenticado (es decir, se garantiza su identidad) mientras que el cliente se mantiene sin autenticar},
    symbol={SSL},
}
\newglossaryentry{ssl-a}{type=\acronymtype, name={SSL}, description={Secure Sockets Layer. \textit{Glosario:} \glslink{ssl}{SSL}}, see=[]{}}

% Sensor
\newglossaryentry{sensor}
{
    name={Sensor},
    text={sensor},
    description={Dispositivo diseñado para detectar información de una magnitud externa y y transformarla en otra magnitud, normalmente eléctrica, que sea posible cuantificar y manipular},
    plural={sensores}
}

% Shell
\newglossaryentry{shell}
{
    name={Shell},
    text={shell},
    description={Intérprete de comandos; Programa que provee una interfaz de usuario para acceder a los servicios del sistema operativo facilitando la forma en que se invocan los distintos programas disponibles en el dispositivo informático},
}

% SBC
\newglossaryentry{sbc}
{
    name={Single Board Computer (SBC)},
    text={Single Board Computer},
    description={Computador de placa única; placa de tamaño reducido que contiene todos o la mayor parte de los componentes de un ordenador: microprocesador, memoria RAM (\textit{Random Access Memory}), dispositivos de entrada/salida (E/S), etc},
    symbol={SBC}
}
\newglossaryentry{sbc-a}{type=\acronymtype, name={SBC}, description={Single Board Computer. \textit{Glosario:} \glslink{sbc}{SBC}}, see=[]{}}

% Sistema de control de versiones 
\newglossaryentry{controlversiones}
{
    name={Sistema de Control de Versiones},
    text={sistema de control de versiones},
    description={Método para controlar y registrar las diferentes versiones por las que pasará un archivo o conjunto de archivos a lo largo del tiempo, de modo que permite la recuperación de versiones específicas en un futuro},
}

% Sistema operativo
\newglossaryentry{so}
{
    name={Sistema Operativo (SO)},
    text={sistema operativo},
    description={Conjunto de programas de un sistema informático encargados de controlar y gestionar los procesos básicos y recursos hardware. Además, permite el funcionamiento de otros programas},
    symbol={SO}
}
\newglossaryentry{so-a}{type=\acronymtype, name={SO}, description={Sistema Operativo. \textit{Glosario:} \glslink{so}{SO}}, see=[]{}}

% SAAS
\newglossaryentry{saas}
{
    name={Software as a Service (SaaS)},
    text={Software as a Service},
    description={\textit{Software} como servicio; modelo de distribución de \textit{software} en el que tanto el \textit{software} como los datos manejados son centralizados y alojados en un servidor externo al usuario y gestionados por el proveedor del servicio por lo que es éste el encargado de garantizar la disponibilidad, seguridad, mantenimiento, soporte, etc. de la herramienta},
    symbol={SaaS}
}
\newglossaryentry{saas-a}{type=\acronymtype, name={SaaS}, description={Software as a Service. \textit{Glosario:} \glslink{saas}{SaaS}}, see=[]{}}	

% SSD
\newglossaryentry{ssd}
{
    name={Solid-State Drive (SSD)},
    text={Solid-State Drive},
    description={Unidad de estado sólido; tipo de dispositivo de almacenamiento de datos que utiliza memoria no volátil para almacenar datos, en lugar de los discos magnéticos de las unidades de discos duros convencionales HDD (\textit{Hard Disk Drive})},
    symbol={SSD}
}
\newglossaryentry{ssd-a}{type=\acronymtype, name={SSD}, description={Solid-State Drive. \textit{Glosario:} \glslink{ssd}{SSD}}, see=[]{}}

% Spring
\newglossaryentry{spring}
{
    name={Spring},
    text={Spring},
    description={\textit{Framework} de código abierto y de propósito general creado por Rod Johnson para el desarrollo de aplicaciones en el entorno Java, principalmente},
}

% Stakeholder
\newglossaryentry{stakeholder}
{
    name={Stakeholder},
    text={stakeholder},
    description={En gestión de proyectos, todas aquellas personas u organizaciones que afectan o son afectadas por el proyecto, ya sea de forma positiva o negativa},
}

% SMBus
\newglossaryentry{smbus}
{
    name={System Management Bus (SMBus)},
    text={System Management Bus},
    description={Bus de Administración del Sistema; subconjunto del protocolo I2C (\textit{Inter-Integrated Circuit})},
    symbol={SMBus}
}
\newglossaryentry{smbus-a}{type=\acronymtype, name={SMBus}, description={System Management Bus. \textit{Glosario:} \glslink{smbus}{SMBus}}, see=[]{}}

% Tensión eléctrica
\newglossaryentry{tension}
{
    name={Tensión eléctrica},
    text={tensión eléctrica},
    description={También conocido como voltaje, es una magnitud física que cuantifica la diferencia de potencial eléctrica entre dos puntos.​ La unidad de resistencia en el Sistema Internacional (SI) es el voltio (V)},
}

% TLS
\newglossaryentry{tls}
{
    name={Transport Layer Security (TLS)},
    text={Transport Layer Security},
    description={Seguridad de la capa de transporte; protocolo criptográfico definido por primera vez en 1999 que garantiza comunicaciones seguras por una red, comúnmente Internet. Se basa en las especificaciones previas de SSL (\textit{Secure Sockets Layer}) por lo que se considera un protocolo más estable y más seguro que éste},
    symbol={TLS},
}
\newglossaryentry{tls-a}{type=\acronymtype, name={TLS}, description={Transport Layer Security. \textit{Glosario:} \glslink{tls}{TLS}}, see=[]{}}

% Token
\newglossaryentry{token}
{
    name={Token},
    text={token},
    description={Cadena de caracteres que tiene un significado coherente en cierto lenguaje de programación. Suelen ser utilizados como identificadores, para el proceso de autenticación, etc},
}

% Transaction
\newglossaryentry{transaction}
{
    name={Transaction},
    text={transaction},
    description={Transacción; concepto relativo a Hyperledger Composer (HC) y la Blockchain (BC) de Hyperledger Fabric (HF) que hace referencia al procedimiento por el cual los participantes (\textit{participants}) interactúan con los activos (\textit{assets}). Este concepto es común a otros tipos de Blockchain (BC)},
}

% UI
\newglossaryentry{ui}
{
    name={User Interface (UI)},
    text={User Interface},
    description={Medio con que el usuario puede comunicarse con una máquina, computadora o dispositivo, y comprende todos los puntos de contacto entre el usuario y el sistema},
    symbol={UI},
}
\newglossaryentry{ui-a}{type=\acronymtype, name={UI}, description={User Interface. \textit{Glosario:} \glslink{ui}{UI}}, see=[]{}}

% UML
\newglossaryentry{uml}
{
    name={Unified Modeling Language (UML)},
    text={Unified Modeling Language},
    description={Lenguaje unificado de modelado; lenguaje de modelado para documentar la arquitectura, el diseño y la implementación de sistemas \textit{software}, tanto en estructura como en comportamiento},
    symbol={UML},
}
\newglossaryentry{uml-a}{type=\acronymtype, name={UML}, description={Unified Modeling Language. \textit{Glosario:} \glslink{uml}{UML}}, see=[]{}}

% Unix
\newglossaryentry{unix}
{
    name={Unix},
    text={Unix},
    description={Sistema operativo portable, multitarea y multiusuario, desarrollado en el año 1969 por un grupo de empleados (Ken Thompson y Dennis Ritchie) de los laboratorios Bell de la compañía AT\&T. Fue creado como un sistema operativo propietario para manejar servidores},
}

% URL
\newglossaryentry{url}
{
    name={Uniform Resource Locator (URL)},
    text={Uniform Resource Locator},
    description={Localizador de recursos uniforme; secuencia de caracteres que sigue un estándar y que permite denominar recursos dentro del entorno de Internet para que puedan ser localizados},
    symbol={URL},
}
\newglossaryentry{url-a}{type=\acronymtype, name={URL}, description={Uniform Resource Locator. \textit{Glosario:} \glslink{url}{URL}}, see=[]{}}

% VCC
\newglossaryentry{vcc}
{
    name={Voltage at the Common Collector (VCC)},
    text={Voltage at the Common Collector},
    description={Voltaje de corriente continua; pin de entrada de potencia principal de un dispositivo que suele ser mayor a 5 voltios (V) en circuitos lógicos típicos. El voltaje es más alto con respecto a la conexión a tierra (\textit{Ground} -GND-)},
    symbol={VCC},
}
\newglossaryentry{vcc-a}{type=\acronymtype, name={VCC}, description={Voltage at the Common Collector. \textit{Glosario:} \glslink{vcc}{VCC}}, see=[]{}}

% VM
\newglossaryentry{vm}
{
    name={Virtual Machine (VM)},
    text={Virtual Machine},
    description={Máquina virtual; \textit{software} que permite emular el funcionamiento de un sistema de computación y ejecutar programas como si fuese un sistema real con la salvedad de que los componentes son virtuales},
    symbol={VM},
}
\newglossaryentry{vm-a}{type=\acronymtype, name={VM}, description={Virtual Machine. \textit{Glosario:} \glslink{vm}{VM}}, see=[]{}}
	
% WAR
\newglossaryentry{war}
{
    name={Web Application Archive (WAR)},
    text={Web Application Archive},
    description={Archivo de aplicación web; archivo JAR (\textit{Java ARchive}) utilizado para distribuir una colección de clases Java,  \textit{servlets}, librerías de etiquetas, páginas web estáticas, etc. que juntos constituyen una aplicación web},
	symbol={WAR},
}
\newglossaryentry{war-a}{type=\acronymtype, name={WAR}, description={Web Application Archive. \textit{Glosario:} \glslink{war}{WAR}}, see=[]{}}

% WiFi
\newglossaryentry{wifi}
{
    name={Wireless Fidelity (WiFi)},
    description={Fidelidad inalámbrica; tecnología que permite la interconexión inalámbrica de dispositivos electrónicos basados en los estándares IEEE 802.11. Inicialmente fue creado para acceder a redes locales inalámbricas, aunque actualmente es utilizado frecuentemente para establecer conexiones a Internet},
    symbol={WiFi},
}
\newglossaryentry{wifi-a}{type=\acronymtype, name={WiFi}, description={Wireless Fidelity. \textit{Glosario:} \glslink{wifi}{WiFi}}, see=[]{}}

% YAML
\newglossaryentry{yaml}
{
    name={YAML Ain't Markup Language (YAML)},
    text={YAML Ain't Markup Language},
    description={Formato estándar, propuesto por Clark Evans en 2001, de serialización de datos legible por los seres humanos para todos los lenguajes de programación},
    symbol={YAML},
}
\newglossaryentry{yaml-a}{type=\acronymtype, name={YAML}, description={YAML Ain't Markup Language. \textit{Glosario:} \glslink{yaml}{YAML}}, see=[]{}}

