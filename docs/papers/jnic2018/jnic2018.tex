% Paper - JNIC 2018

% Commands
\newcommand{\CLASSINPUTinnersidemargin}{18mm}
\newcommand{\CLASSINPUToutersidemargin}{12mm}
\newcommand{\CLASSINPUTtoptextmargin}{20mm}
\newcommand{\CLASSINPUTbottomtextmargin}{25mm}
\documentclass[10pt,conference,a4paper]{IEEEtran}

% Packages
\usepackage[utf8]{inputenc}
\usepackage{times}
\usepackage{graphicx}
\usepackage{subfigure}
\DeclareGraphicsExtensions{.png,.eps,.ps,.pdf}
\usepackage{url}
\hyphenation{si-guien-do}
\usepackage[spanish,es-tabla]{babel}
\usepackage{hyperref}

\begin{document}

% Title
\title{Desarrollo de un sistema de trazabilidad en entornos IoT mediante Hyperledger}

% Authors
\author{\IEEEauthorblockN{Jesús Iglesias García}
\IEEEauthorblockA{Escuela Politécnica Superior\\
Departamento de Ingeniería Informática \\
Universidad Autónoma de Madrid\\
jesusgiglesias@gmail.com}
\and
\IEEEauthorblockN{David Arroyo Guardeño}
\IEEEauthorblockA{Escuela Politécnica Superior\\
Departamento de Ingeniería Informática\\
Universidad Autónoma de Madrid \\
david.arroyo@uam.es}}

\maketitle

% Abstract
\begin{abstract}
    En la actualidad un grueso importante de empresas tienen un
    interés especial en construir una nueva generación de aplicaciones
    transaccionales que establezcan confianza, responsabilidad y
    transparencia en su núcleo junto con una arquitectura abierta y
    distribuida. Aquí es de especial relevancia el sistema de pago de
    gestión no centralizado Bitcoin \cite{JNIC2018_3}, que está basado
    en la tecnología Blockchain (BC) la cual habilita la creación de un
    sistema inmutable de registro de eventos. Desde su origen ha
    propiciado todo un conjunto de iniciativas que, prescindiendo de la
    existencia de una tercera parte de confianza, puedan
    proporcionar protocolos para la protección de la integridad de la
    información.

    Una de estas iniciativas es Hyperledger \cite{JNIC2018_4}
    -\textit{standard open source} de BC-, se centra en apoyar
    este tipo de transacciones de negocio para mejorar numerosos
    aspectos de rendimiento y empresariales. Hyperledger es un ejemplo
    de BC \emph{permisionada}, ya que la inserción de
    información requiere que las entidades sean previamente
    autenticadas. Este modelo de control de acceso es de alto interés
    en el contexto de la trazabilidad de recursos y productos de una
    organización. El presente proyecto aborda el desarrollo de una
    prueba de concepto para implementar Hyperledger en el Internet de
    las Cosas (\emph{Internet of Things} -IoT-). El objetivo será la recolección de eventos mediante
    sensores situados en una Raspberry Pi, y la inclusión de
    incidencias en Hyperledger. Asimismo, la información introducida
    podrá ser consumida en tiempo real mediante un cliente web.
\end{abstract}

% Keywords
\begin{IEEEkeywords}
Hyperledger, Blockchain, Seguridad, Privacidad, Internet de las Cosas
\end{IEEEkeywords}

{\bf Tipo de contribución:}  {\it Investigación en desarrollo}

% Section
\section{Introducción}

Internet ha cambiado la forma de vida y la sociedad en
general. Nuestra actividad diaria depende cada vez más de la
información que obtenemos de Internet, de forma que necesitamos contar
con mecanismos que nos permitan dirimir si los datos que obtenemos son
fiables. Esta cuestión se ha resuelto tradicionalmente a través de
alguna suerte de Tercera Parte Confiable (TPC). Sin embargo, la
participación de intermediarios también presenta desventajas, entre
las que se encuentra la posible degradación de nuestra privacidad si 
la TPC  accede sin permiso a nuestra información sensible y personal.

Una posible solución es la descentralización de la gestión de la
información, de
forma que no se necesita una autoridad central intermediaria que tenga
acceso a los datos. Aquí, es donde nace el
concepto de cadena de bloques o \textit{blockchain}, un libro de
registros o transacciones (\textit{ledger}) distribuido, del que todos
los que participan en la red almacenan una copia que se actualiza
mediante un protocolo \textit{Peer-to-Peer} de consenso. En este contexto, el
protocolo de consenso distribuido se constituye en garante de la
integridad de la información y, por ende, de su veracidad. 

En su origen, BC se inventó para sustentar Bitcoin, la primera
criptomoneda descentralizada no emitida por un banco central. Sin
embargo, su aplicación no queda limitada a las criptomonedas. En
efecto, la BC de Bitcoin habilita la escritura de información no
vinculada a transacciones (por ejemplo, mediante el campo
\textsc{\textit{OP\_RETURN}}). Esta información puede ser utilizada como canal
de control de trazabilidad de las cadenas de
producción (por ejemplo,
\href{https://www.everledger.io/}{Everledger}), de certificación de
documentación (como es el caso de
\href{https://stampery.com/}{Stampery} y
\href{https://maidsafe.net/}{MaidSafe}), de hipotecas (e.g,
\href{http://www.zensar.com/blogs/2017/05/reinventing-mortgage-with-blockchain/}{Zensar}),
de títulos o cualquier otro documento oficial
(\href{http://bitfury.com/}{Bitfury},
\href{https://www.factom.com/}{Factom},
\href{https://chromaway.com/}{ChromaWay} y
\href{https://www.velox.re/}{Velox.re} son ejemplos de
\emph{start-ups} que ofrecen este tipo de servicio), así como
aplicaciones de control de integridad de información en ámbitos
relacionados con la seguridad lógica\footnote{Aquí cabe destacar el
  proyecto \href{https://guardtime.com/}{Guardtime}, que cuenta con el
  patrocinio de \href{https://www.darpa.mil/}{DARPA}.}

Uno de los ámbitos donde BC tiene especial interés es en IoT. La IoT
constituye una red de dispositivos físicos que por naturaleza se
conectan entre sí e intercambian datos para hacer nuestras vidas más
sencillas y eficientes. Sin embargo, cada dispositivo puede ser
cualquier \textit{cosa} desde una televisión, un vehículo, un
aspirador o hasta un frigorífico y todos funcionan de forma diferente
y con diferentes niveles de seguridad implementados. Esta variabilidad
en las interfaces de acceso a la información y los mecanismos de
intercambios de datos, introduce una incertidumbre en lo referente a
las diversas fuentes de datos. Aquí es donde entraría en juego la BC,
en específico aquellas evoluciones de la BC de Bitcoin mediante la
incorporación de los denominados \emph{smart
  contracts}\footnote{Código informático que se ejecuta en la
  blockchain y hace cumplir un contrato de manera automática. Este
  concepto fue introducido por Nick Szabo en 1996 (ver \cite{JNIC2018_6},
  último acceso 08/03/2018).} y
de modelos de control de acceso.

% Section
\section{Hyperledger}

Hyperledger es un proyecto de código abierto y de carácter
colaborativo anunciado en el año 2015 por la fundación Linux que
implementa la tecnología BC de uso privado y que está orientada al uso
empresarial gracias a su capacidad de ejecución de transacciones
privadas. Pretende ser una plataforma multidisciplinar, es decir,
posibilita la creación de \textit{chaincodes} (los \textit{smart
  contracts} de Hyperledger) en cualquier lenguaje, así como la
elección del método de consenso a realizar. Dentro de los proyectos de
este consorcio, el más conocido es la plataforma de BC permisionada:
Hyperledger Fabric. Esta plataforma presenta una arquitectura modular
y características de identidad y seguridad y está orientada al
desarrollo de aplicaciones empresariales.

% Section
\section{Hyot}

Hyot es la prueba de concepto o \textit{Proof of Concept} (PoC)
implementada para la trazabilidad de un entorno controlado de IoT
mediante la tecnología Hyperledger. Si bien cabe decir que es
fácilmente escalable a ampliar un mayor rango de cobertura en cuanto a
IoT se refiere. Esta solución gestiona de forma transparente para el
usuario una serie de sucesos -temperatura, humedad, distancia, etc.-
que son monitorizados desde sensores conectados a un ordenador de
placa reducida (\textit{Single Board Computer}) como es la Raspberry
Pi3. En caso de que se produzca una incidencia, las lecturas de los
sensores son almacenados en una base de datos. Una incidencia no es
sino una acción no controlada que tiene lugar en el entorno que se
está vigilando, y origina la ejecución de un protocolo de alerta que
comprende: (1) la captura de un vídeo a través de una cámara conectada
a la Raspberry Pi;(2) almacenamiento en la nube del vídeo. En la
medida que estos servicios de almacenamiento son una TPC que pueden
poner en riesgo la privacidad de las evidencias, hemos asumido un
modelo de confianza nula en el cual todos la información es cifrada
mediante  GPG (\textit{GNU Privacy Guard}) antes de ser almacenada en
al nube.

El  punto central de esta PoC es garantizar que el
registro de un  suceso anómalo no ha sido indebidamente
modificado, de forma que una vez registrada una incidencia se tenga
total certeza respecto a su integridad y veracidad. Este objetivo se
consigue con la tecnología Hyperledger Fabric, ya que proporciona un
protocolo de consenso distribuido para la protección de integridad y
un control de acceso que permite identificar a los agentes que
introducen datos en la BC. En el dominio de nuestro caso de uso, en Hyperledger
Fabric se establecen las transacciones posibles a ejecutar junto con
los activos (\textit{assets}): marca temporal (\textit{timestamp}) en
la que ocurrió el suceso, el \textit{hash} del contenido del vídeo
encriptado -calculado con el algoritmo criptográfico SHA3- e incluso
los valores de los sensores medidos.

Además, en este tipo de sistemas donde se controla un entorno (ya sea
por cuestiones de seguridad o por cualquier otro motivo), es importante
informar al administrador del sistema de que un suceso no
controlado está sucediendo. Esto se consigue con la notificación de la
información actual mediante un \textit{email} a la dirección de correo
electrónico configurada.

A modo resumen, se puede detallar que Hyot se compone de tres partes:

\begin{itemize}
  \item Script de monitorización de los sucesos de los sensores de una
    Raspberry Pi3.
  \item Protocolo de registro de incidencias en la BC de Hyperledger
    Fabric y almacenamiento de evidencias en la nube. 
  \item Sistema web que actúa como cliente y consume la
          información en tiempo real registrada por la Raspberry Pi 3.
\end{itemize}

% Section
\section{Conclusiones}

La rápida evolución del mercado del IoT ha provocado una explosión en
el número y la variedad de soluciones IoT, lo que ha creado grandes
desafíos a medida que la industria evoluciona, principalmente, la
necesidad urgente de un modelo IoT seguro para realizar tareas comunes
como detección, almacenamiento y comunicación. Es por ello que la
aplicación de la tecnología BC puede favorecer el despliegue de
soluciones y arquitecturas más seguras.

Con este objetivo surge Hyot, un proyecto software de código
abierto\footnote{El código fuente será liberado en Github una vez
  finalizado y presentado el Trabajo Fin de Máster.} que propone una
prueba de concepto simple y novedosa sobre conceptos y tecnologías en
auge como la BC. Con ello, se pretende mostrar las ventajas de las
arquitecturas basadas en BC a la hora de diseñar e implementar
protocolos de control y auditoría en IoT.

Por último, cabe mencionar que se trata de un proyecto en pleno
desarrollo al momento de redactar este documento y que será presentado
como Trabajo Fin de Máster (TFM) en el \textit{Máster de Ingeniería
  Informática} de la Escuela Politécnica Superior de la Universidad
Autónoma de Madrid (UAM). El ciclo de vida de este proyecto no pretende ser el
habitual de un trabajo de universidad, y la idea principal es continuar
con su desarrollo una vez presentado dado que se engloba dentro de un
área con  gran interés tanto para empresas nacionales como
internacionales. Como parte del trabajo futuro se incorporarán nuevas
funcionalidades relacionadas con la gestión de la identidad y su
equilibrio con la protección de la privacidad\footnote{Aquí se tendrá
  en mente aplicar los esfuerzos para incluir el entorno Idemix en
  Hyperledger Fabric.}, con la interoperabilidad con BC
públicas\footnote{En la línea marcada por la iniciativa Hyperledger
  Sawtooth, \cite{JNIC2018_5}
  (Último acceso 08/03/2018).}, así como con la incorporación de técnicas de
aprendizaje automático  que permitan identificar información sensible
y relevante como evidencia digital. 
 
% Section
\section*{Agradecimientos}

Este trabajo ha sido financiado a través de los proyectos
CIBERDINE (S2013/ICE-3095) -Comunidad Autónoma de Madrid- y
MINECO/FEDER DPI2015-65833-P (Gobierno de España).

% Bibliography
\bibliographystyle{IEEEtran}

\begin{thebibliography}{99}
\bibitem{JNIC2018_1}
Ali Dorri, Salil Kanhere, Raja Jurdak: `Towards an Optimized BlockChain for IoT`, 2017.
\bibitem{JNIC2018_7}
Ethereum White Paper

https://github.com/ethereum/wiki/wiki/White-Paper
\bibitem{JNIC2018_6}
Nick Szabo: `Formalizing and Securing Relationships on Public Networks`, First Monday, vol.2, n.9, 1997.

http://journals.uic.edu/ojs/index.php/fm/article/view/548
\bibitem{JNIC2018_4}
Hyperledger.

https://www.hyperledger.org
\bibitem{JNIC2018_5}
Hyperledger Sawtooth.

https://www.hyperledger.org/projects/sawtooth
\bibitem{JNIC2018_2}
Marko Vukolic: `Hyperledger Fabric - An Open-Source Distributed Operating System for Permissioned Blockchains`, Swiss Blockchain Summer School Lausann, 2017.
\bibitem{JNIC2018_3}
Pedro Franco: `Understanding Bitcoin: Cryptography, Engineering and Economics`, Wiley Finance Series, 2014.

\end{thebibliography}

\end{document}
